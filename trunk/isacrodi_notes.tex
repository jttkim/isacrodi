\documentclass[a4paper,fleqn]{article}

\usepackage[T1]{fontenc}
\usepackage{times}
\usepackage[english]{babel}
\usepackage{din_a4}
\usepackage{graphicx}
\usepackage{caption}

\renewcommand{\captionfont}{\footnotesize\bfseries}

\newcommand{\computercode}[1]{\texttt{#1}}
\newcommand{\computermeta}[1]{\texttt{\textbf{\textit{#1}}}}

\parindent0pt
\parskip 1ex plus 0.3ex minus 0.2ex


\begin{document}

\title{GRNInfo Design Notes}
\author{}
\maketitle

\includegraphics[width=10cm]{isacrodi_usecasediagram.eps}

\includegraphics[width=10cm]{isacrodi_classdiagram.eps}

\pagebreak

\appendix



\includegraphics[width=10cm]{isacrodi_scratch_classificationflow.eps}

\section{Requirements}

The system has to be able to recognise when it cannot classify the
disorder underpinning a CDR.


\section{Coding Standards}

Suggestions:

\begin{itemize}

\item Two spaces per indentation level.

\item Open braces on separate line.

\item White space around binary operators.

\item No white space between unary operators and operand.

\item CamelCase identifiers, class names start with uppercase
  character, method and variable names start with lowercase character.

\item Full Javadoc documentation, minimally consisting of one sentence
  in the main block and documentation of all of parameters and of the
  return value.

\end{itemize}

\end{document}


%%% Local Variables: 
%%% mode: latex
%%% TeX-master: t
%%% End: 
