\documentclass[a4paper,fleqn]{article}

\usepackage[T1]{fontenc}
\usepackage{times}
\usepackage[english]{babel}
\usepackage{din_a4}
\usepackage{graphicx}
\usepackage{caption}

\renewcommand{\captionfont}{\footnotesize\bfseries}

\newcommand{\computercode}[1]{\texttt{#1}}
\newcommand{\computermeta}[1]{\texttt{\textbf{\textit{#1}}}}

\parindent0pt
\parskip 1ex plus 0.3ex minus 0.2ex


\begin{document}

\title{GRNInfo Design Notes}
\author{}
\maketitle

\section{Design}

\subsection{Use Cases}

\includegraphics[width=10cm]{isacrodi_usecasediagram.eps}

\begin{description}
\item[Create CDR:] A user requests creation of a new CDR, and the
  system sets up one for them (asking for a name of the CDR etc.)
\item[Build CDR:] A user adds or amends further items of information
  to the CDR.
\item[request and obtain diagnosis:] The user requests diagnosis of
  the case described by a CDR. The system computes a diagnosis and
  reports that back to the user. The diagnosis is comprised of a list
  of disorders that may affect the crop, and a list of recommendations
  for treating the crop. The diagnosis may identify one (or more?)
  descriptors that would be informative for improving precision of the
  diagnosis (e.g.\ to distinguish diseases).
\item[identify possible disorders:] The system computes probabilities
  (or other scores) for each disease stored in the database. It
  provides some indication which disorders are likely, and thus it is
  a possible result that no likely disorders have been found.
\end{description}


\includegraphics[width=10cm]{isacrodi_classdiagram.eps}

\pagebreak

\appendix



\includegraphics[width=10cm]{isacrodi_scratch_classificationflow.eps}

\section{Requirements}

The system has to be able to recognise when it cannot classify the
disorder underpinning a CDR.


\section{Coding Standards}



\subsection{EJB Standards}

\subsubsection{Entities}

\begin{itemize}

\item Entities have an \computercode{Integer} id, which is always
  called \computercode{id}. Deviations are explained in the Javadoc
  documentation.

  The use of \computercode{Integer} allows
  the \computercode{id} to take on the value \computercode{null}, to
  signify entity instances that have not yet been persisted or are
  otherwise transient.

\item Entities in the \computercode{isacrodi.ejb.entity} package.  

\item No specification of table or column names via annotation (the
  names automatically provided / generated by the EJB container should
  be used).

\item Column annotation on accessors and mutators, not on instance
  variables.

\end{itemize}


\subsection{Code Format}

\begin{itemize}

\item Two spaces per indentation level.

\item Open braces on separate line.

\item White space around binary operators.

\item No white space between unary operators and operand.

\item CamelCase identifiers, class names start with uppercase
  character, method and variable names start with lowercase character.

\item Full Javadoc documentation, minimally consisting of one sentence
  in the main block and documentation of all of parameters and of the
  return value.

\end{itemize}

\end{document}


%%% Local Variables: 
%%% mode: latex
%%% TeX-master: t
%%% End: 
