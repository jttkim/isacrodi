\documentclass[a4paper,fleqn]{article}

\usepackage[T1]{fontenc}
\usepackage{times}
\usepackage[english]{babel}
%\usepackage{din_a4}
\usepackage{graphicx}
\usepackage{caption}
\usepackage{dsfont}


\renewcommand{\captionfont}{\footnotesize\bfseries}

\newcommand{\computercode}[1]{\texttt{#1}}
\newcommand{\computermeta}[1]{\texttt{\textbf{\textit{#1}}}}

\parindent0pt
\parskip 1ex plus 0.3ex minus 0.2ex


\begin{document}

\title{GRNInfo Design Notes}
\author{}
\maketitle

\section{Design}

\subsection{Use Cases}

\includegraphics[width=10cm]{isacrodi_usecasediagram.eps}

\begin{description}
\item[Create CDR:] A user requests creation of a new CDR, and the
  system sets up one for them (asking for a name of the CDR etc.)
\item[Build CDR:] A user adds or amends further items of information
  to the CDR.
\item[request and obtain diagnosis:] The user requests diagnosis of
  the case described by a CDR. The system computes a diagnosis and
  reports that back to the user. The diagnosis is comprised of a list
  of disorders that may affect the crop, and a list of recommendations
  for treating the crop. The diagnosis may identify one (or more?)
  descriptors that would be informative for improving precision of the
  diagnosis (e.g.\ to distinguish diseases).
\item[identify possible disorders:] The system computes probabilities
  (or other scores) for each disease stored in the database. It
  provides some indication which disorders are likely, and thus it is
  a possible result that no likely disorders have been found.
\end{description}


\includegraphics[width=10cm]{isacrodi_classdiagram.eps}

\pagebreak

\appendix


\section{Classification}


\includegraphics[width=10cm]{classification_diagram.eps}

\includegraphics[width=10cm]{isacrodi_scratch_classificationflow.eps}




\section{Requirements}

The system has to be able to recognise when it cannot classify the
disorder underpinning a CDR.


\section{Implementation Notes}

\subsection{Converting Feature Vectors to Formats Required by
  Classifiers}

Isacrodi aims to provide a generic feature vector concept in which
components of multiple (polymorphic) type (categorial, ordinal and
numeric) can be used, and components can be missing, or perhaps take
on a ``not available'' value, denoted by NA henceforth.\footnote{This
  is not fully implemented at this stage, but that should not cause
  introduction of any unnecessarily tight coupling to classifiers.}
However, classifiers such as Support Vector Machines typically require
samples to be presented as vectors (arrays of \computercode{double} or
similar data structures), or as sparse vectors (effectively
$\mathds{N}_{\ge 0} \rightarrow \mathds{R}$ maps). This necessitates
some form of mapping from Isacrodi level feature vectors to the vector
format required by the classifier. The following is an outline
specification of this mapping:
\begin{itemize}
\item The mapping is constructed based on the training data, before
  the classifier is actually trained. The same mapping is then used to
  map the Isacrodi level feature vectors to classifier level vectors
  during training and during subsequent classification operations (aka
  ``predictions'').
\item The category of each component is expressly stated to the
  constructor of the mapper. (Note: The mapper might be able to infer
  types from the dynamic type of components, but this requires that
  for each Isacrodi level component, there is at least one sample with
  a non-\computercode{null} (i.e.\ non-NA) value in the training set,
  and also results in a plethora of illegal states (where the type of
  a component is not the same in all samples) that the constructor
  would have to check for. It therefore makes sense to require that
  the user (the expert who trains the classifier) states the expected
  type for each component.)
\item For mapping to plain vectors (of constant dimension):
  \begin{itemize}
  \item Each Isacrodi level numeric component is mapped to two
    classifier level components, the actual value and the presence
    indicator.  present. A NA value is mapped to the mean of all
    non-NA values in the training set as the actual value, and $0.0$
    as the presence indicator. A non-NA value is mapped to the value
    provided as the actual value and $1.0$ as the presence indicator.
  \item For each Isacrodi level categorial component, the set of
    states is taken from the training set. For each state, one
    component is designated in the classifier level vector, and
    additionally a presence indicator is designated. For a NA value,
    all components are mapped to $0.0$. For a non-NA, value, the
    component corresponding to the state and the presence indicator
    are $1.0$, and all other components are $0.0$.
  \item (For ordinal values, the order needs to be specified. This
    could be done externally or by implementing the
    \computercode{Comparable} interface for the types. As we do not
    have a design for ordinal components, specification of the mapping
    technique is deferred until this design is drawn up.)
  \end{itemize}
\item To be confirmed: For mapping to sparse vectors, the presence
  indicator may be omitted. Other mapping takes place as specified for
  plain vectors above.
\item A central responsibility of the mapper is to designate indexes
  for the classifier level components upon construction and to ensure
  that all mapping takes place using the same indexes.
\item The mapper provides two strategies for dealing with components
  or states (in the case of categorial components) that are unknown:
  \begin{itemize}
  \item Strict: An exception is thrown upon encountering an unknown
    component or state.
  \item Lenient: Unknown components, or known categorial components
    containing unknown states, are ignored.
  \end{itemize}
\end{itemize}


\subsection{The CRUD System}

The Struts action \computercode{CrudAction} (CRUD = ``Create, Read,
Update, Delete'') provides a simple ``database browser'' like web
interface to the Isacrodi entity set. The system can display lists of
links to all entities to a given type and details on entities. A
detail page contains links to all associated entities, and all
properties in textual form.

The CRUD system mostly uses reflection but requires that entities
implement the \computercode{IsacrodiEntity} interface.

For deleting entities, the system invokes the \computercode{unlink}
method to remove all associations of the entity to others.

\section{Coding Standards}


\subsection{General Java}

\subsubsection{Code Organisation}

\begin{itemize}

\item Methods order should be consistent with the partial ordering
  implied by ther caller / callee relationship, i.e.\ a method should
  be declared after the declaration of all the methods it
  calls. Exceptions to this are permitted only where necessary to use
  indirect recursion.

\end{itemize}


\subsubsection{Code Format}

\begin{itemize}

\item Two spaces per indentation level.

\item Open braces on separate line.

\item White space around binary operators.

\item No white space between unary operators and operand.

\item CamelCase identifiers, class names start with uppercase
  character, method and variable names start with lowercase character.

\item Instance variables are always prefixed with \computercode{this}
  (i.e.\ no use of implicit instance variables).

\item Full Javadoc documentation, minimally consisting of one sentence
  in the main block and documentation of all of parameters and of the
  return value.

\end{itemize}


\subsection{EJB Standards}

\subsubsection{Entities}

\begin{itemize}

\item Entities have an \computercode{Integer} id, which is always
  called \computercode{id}. Deviations are explained in the Javadoc
  documentation.

  The use of \computercode{Integer} allows
  the \computercode{id} to take on the value \computercode{null}, to
  signify entity instances that have not yet been persisted or are
  otherwise transient.

\item Entities in the \computercode{isacrodi.ejb.entity} package.

\item Properties mapping relationships are called after the class
  being referenced, with the first letter lowercased. Exceptions are
  made only where there are multiple relationships between the same
  two entities.

\item All entities are serializable, so that they can be passed out to the
  web tier as model instances.

\item All entities provide a version column (for optimistic locking).

\item Properties mapping to-many relationships are called
  \computercode{\computermeta{propertyname}Set}.

\item No specification of table or column names via annotation (the
  names automatically provided / generated by the EJB container should
  be used).

\item Column annotation on accessors and mutators, not on instance
  variables.

\item Constructors are set out in a cascaded way, such that each
  constructor invocation leads to an explicit superclass constructor
  invocation.

\item All sets mapping to-many relationships are initialised in the
  default constructor using \computercode{HashSet}s. This ensures that
  a usable set is always available, even when constructing instances
  outside of an EJB container, and thus facilitates testing.

\item For each to-many relationship, entities should implement an
  \computercode{add\computermeta{Property}} method that adds a
  relationship. Where the mapping is bidirectional, it is the
  responsibility of this ``adder'' method to set up both directions.

  For one-to-one relationships, both participating entity classes
  should implement a \computercode{link\computermeta{Property}} method
  that sets up both directions.

  Notice that bidirectional linking must not be implemented by any
  ``setter'' method, as these are defined bean interfaces that are
  used by the container.

\end{itemize}


\section{Issues}

\begin{itemize}

\item The struts plugin contains an \computercode{xwork-core} jar
  which contains a \computercode{javassist} package. That clashes with
  the \computercode{javassist} package included with JBoss. According
  to Apache JIRA report
  WW-3308\footnote{\computercode{https://issues.apache.org/jira/browse/WW-3308}}
  this will be removed in struts 2.2.0.

  Possible solutions:
  \begin{itemize}
  \item Wait for struts 2.2.0.
  \item Install a local maven repository that provides struts 2.2.0.
  \item Remove \computercode{javassist} from the current
    \computercode{xwork-core} jar and pack that into the war (tested).
  \end{itemize}

\end{itemize}


\section{To Do List}

\begin{itemize}

\item Move \computercode{org.isacrodi.diagnosis} into separate Maven
  module.

\end{itemize}


\section{Unsorted notes}

\section{Monteria Corpoica Notes}

\begin{itemize}

\item For values of categorical descriptors use illustrative sketches.
  Example: Illustrations of distribution of affected plants within a
  field, such as patchy, scattered etc.

  The values acceptable for categories thus illustrated will have to
  be controlled.

\item If geographical location and time of incident are known, weather
  conditions (possibly for the entire season) can be retrieved from
  appropriate services and be used to enhance diagnosis. Not realitic
  within Royal Society project, though.

\item Present images and other descriptive information on the highest
  scoring disorders. The database will need to provide mor descriptive
  information, including images, to support this function.

\item How to deal with agrochemicals etc.\ that may be multiply
  applied?

\item Additional numeric type for holding ranges / minimum / maximum /
  average etc.?

\item Differentiate between ``chemical'' and ``biological'' pest
  control.

\item Recommendations may have to depend on geographical location /
  climate etc., and also greenhouse / open field, level of damage
  (higher levels may suggest more drastic procedures)

\item Multiple values for categorical descriptors should be possible
  for defined descriptors, e.g.\ affected plant part, where multiple
  parts can be affected.

\item Distinguish signs of pest activity from symptoms: symptoms cause
  substantial damage to crops, signs just indicate presence of pests,
  possibly at levels that do not matter.

\item Establish that disorders are really important.

\item Action thresholds decide whether / which procedures should be
  recommended.

\item Include preventative procedures / measures -- mention in crop
  disorder descriptions.

\item Information on similar problems in area -- isolated or
  widespread problem?

\item Key pests vs.\ secondary pest. Key pests need to be managed.

\item hormoligosis.

\end{itemize}


\subsection{UEA Norwich Notes}

\begin{itemize}

\item Consider including specific strain / variety of crop, rather
  than just species name? The variety is important for diagnosing
  disorders (e.g.\ a transgenic BT corn should not be affected by corn
  borers etc.). On the other hand, CDRs pertaining to a large number
  of varietis would need to be collected in order to make use of the
  potential gain in information.

\item Recommendations can be refined based on information contained in
  CDRs. For example, where a crop is ready for harvest and thus has a
  limited life expectancy anyway, recommendations should focus on
  preventing propagation of diagnosed disorders to future crops,
  rather than on protecting the current crop. As another example, if
  the environmental conditions are adverse to a disorder anyway, there
  may be less reason to control that disorder, e.g.\ if the season is
  rainy, controlling insects may be less urgent.

\item Consider including geographical location as a descriptor. This
  could be done in terms of latitude and longitude. At the UI level,
  users can then be supported by finding the latitude / longitude by
  country / city / region.

\end{itemize}


\end{document}


%%% Local Variables:
%%% mode: latex
%%% TeX-master: t
%%% End:
